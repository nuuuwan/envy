\section{Many-to-Many Envy}

Similarly we can define the collective envy felt by all people in the population as $E_{\cdot,\cdot}$ or simply $E$ as follows:

\begin{equation}
    E = \frac{\sum_{i} E_{i, \cdot}}{N} 
    = \frac{\sum_{i} \sum_{j} E_{i,j}}{N^2}
\end{equation}

Let ${p_I}$ denote the proportion of people in the population with income $I$, and let $E_{I, J}$ denote the Envy felt by a person with income $I$ for a person with income $J$.

Then we can write the above as:

\begin{equation}
    E = \frac{\sum_{I} \sum_{J} Np_I  \cdot Np_J  \cdot E_{I,J}}{N^2} = \sum_{I} \sum_{J} p_I  \cdot p_J  \cdot E_{I,J}
\end{equation}

The last representation implies that our definition is invariant to scaling the population. I.e., if we double the population proportionally, the Envy remains the same.

Going forward, we will use this representation of \textit{groups of people with the same income}, rather than \textit{individuals}.



