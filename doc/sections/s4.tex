\section {Minimizing and Maximizing Envy}

\subsection {Perfect Equality}

Trivially, when the entire population has the same income, there is no Envy. 

\begin{equation}
    I_i = k \forall i \implies E = 0
\end{equation}


\subsection {Isolated Classes}

Also, if the population is split into groups where within each groups individuals have the same income, and the relative income across two groups is more than $\alpha^2$, then there is no Envy.

For example, if these incomes are $k, k\gamma^2, k\gamma^4...,$, where $\gamma \ge \alpha$, then there is no Envy.

\begin{equation}
    E = 0
\end{equation}

\subsection {Simillar Classes}

Conversely, if $\gamma < \alpha$, the classes have an overlap in Envy and the ${E}$ is no longer zero.

\subsection {Progressive Classes}

When $\gamma = \alpha$, the classes are progressive and the Envy is maximized. \footnote{This needs to be proved.}